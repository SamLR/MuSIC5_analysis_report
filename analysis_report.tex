%
%  analysis_report
%
%  Created by Sam Cook on 2012-11-06.
%  Copyright (c) 2012 . All rights reserved.
%
\documentclass[]{article}

% Use utf-8 encoding for foreign characters
\usepackage[utf8]{inputenc}

% Setup for fullpage use
\usepackage{fullpage}

% Uncomment some of the following if you use the features
%
% Running Headers and footers
%\usepackage{fancyhdr}

% Multipart figures
%\usepackage{subfigure}

% More symbols
\usepackage{amsmath}
%\usepackage{amssymb}
%\usepackage{latexsym}

% Surround parts of graphics with box
\usepackage{boxedminipage}

% Package for including code in the document
\usepackage{listings}

% If you want to generate a toc for each chapter (use with book)
\usepackage{minitoc}

% This is now the recommended way for checking for PDFLaTeX:
\usepackage{ifpdf}

%\newif\ifpdf
%\ifx\pdfoutput\undefined
%\pdffalse % we are not running PDFLaTeX
%\else
%\pdfoutput=1 % we are running PDFLaTeX
%\pdftrue
%\fi

\newcommand{\nth}[1]{#1\(^\text{th}\)}
\newcommand{\nthTwo}[2]{\(#1^\text{#2}\)}

\ifpdf
\usepackage[pdftex]{graphicx}
\else
\usepackage{graphicx}
\fi
\title{MuSIC 5 Analsis Report}
\author{Sam Cook}

\date{2012-11-06}

\begin{document}

\ifpdf
\DeclareGraphicsExtensions{.pdf, .jpg, .tif}
\else
\DeclareGraphicsExtensions{.eps, .jpg}
\fi

\maketitle


\begin{abstract}
	The \nth{5} MuSIC beamtime (\nth{18} to the \nthTwo{22}{nd} June 2012) was inteded to test the momentum distribution of the muons beam. Here we will discuss the strategy employed to analyse the data and the problems encountered.
\end{abstract}

\section{Experimental set-up}
The aim of MuSIC~5 was to measure the muon momentum distribution at the end of 36\(^{\circ}\) of beampipe. This measurement was made by counting muon decays between two counters (upstream and downstream) and using a degrader and stopping target to select muons of a given momentum. 

By varying the degrader's thickness different regions of the muon momentum distribution could be removed whilst the stopping target limited the the maximum (average, degrader adjusted) momentum that would decay. The total effect of this combination was to select bands of muons of different average momentum. The distribution of momentums could then be inferred by counting the number of muon decays with the specific combination of degrader and stopping target.

The upstream counter consisted of an array of 8 thin (1~mm) plastic scintillators whilst the downstream counter was 5 thicker (3.5~mm) scintillator both upstream and downstream counters were wrapped with mylar and black-wrap.

The counters were read out using MPPCs attached to either end of wavelength-shifting fibres mounted on the scintillators. Each MPPC was attached to a simple filter and had its voltage tuned to the specified value via a voltage divider. The signals from the MPPCs were amplified before being passed to the DAQ logic for digitisation.

\section{Data Acquisition (DAQ)}
The trigger for the system was the detection of a muon in the upstream counter with nothing in the downstream counter i.e. that a muon had stopped in the stopping target. Once a trigger had been established all subsequent hits in either up or downstream counters were recorded on a Multi-TDC.

In addition to TDC readings the amount of charge on each channel (summed over both MPPCs) at the time of trigger was 

Several sets of data were taken with the main differnce between runs being the width of the degrader used (0, 0.5, 1 and 5~mm) although there were also other differences (for example the proton beam current). 

The data from the run ()

\section{Simulation}
\section{Framework}
\section{Algorithm}
\section{Results}
\section{Conclusion}
It's all fucked

\bibliographystyle{plain}
\bibliography{}
\end{document}
